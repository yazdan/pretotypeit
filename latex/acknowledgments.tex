مفهوم پیش‌نمونه سازی و این کتاب بدون تشویق‌ها و پشتیبانی پاتریک کوپ لند
(مدیر و مربی من در گوگل) امکان پذیر نبود. پاتریک نه تنها به توسعه و
بهبود این ایده کمک کرد بلکه مطمئن شد که آنچه را آموخته‌ام به کار خواهم
برد و ایده‌های جدید زود و بصورت معمول عملیاتی خواهم کرد. او همچنین در
انتشار این مفهوم کمک رساند. او سخنرانی‌‌های کلیدی زیاد و موفقی در زمینه
پیش‌نمونه‌سازی در کنفرانس‌های بزرگ در سرتاسر دنیا داشته است.

من بسیار خوش‌شانس بوده‌ام که دو مبدع بزرگ در گوگل با من همکاری
می‌کرده‌اند. استفن اولر و باب اوانز. استفن یک پیش‌نمونه‌سازی بالفطره است
که در توسعه \lr{PretoGen} که ابزاری برای ساختن پیش‌نمونه‌هاست، کمکهای
بسیاری رسانده است. باب یکی از باهوش‌ترین افرادی است که من می‌شناسم است و
منبع الهام بخش من و همچنین چالشگر این پیش‌نمونه سازی در روزهای اولیه آن
بوده است. ایده‌ی پیش‌نمونه سازی وقتی از میان بحث‌های ما در زمان کار
بوجود آمد.

فرد کلیدی دیگری در توسعه، بهبود و عمومی سازی پیش‌نمونه سازی جرمی کلارک
است. او یک متفکر پیش‌رو در حوزه ابداع و بنیان‌گذاری شرکت \lr{FXX} است.
جرمی و من کار بر روی پیش‌نمونه‌سازی را ادامه داده و معمولا ارائه‌های
مشترکی در این زمینه ارائه می‌دهیم.

کارلو آلبرتو پراتزی که استاد بازاریابی دانشگاه \lr{Roma} \lr{Tre} و
بنیانگذاری آزمایشگاه‌های \lr{InnovAction} در ایتالیا است، نه تنها منبع
الهام و نمونه‌های واقعی بوده است بلکه در زمینه اجرای پیش‌نمونه‌سازی در
اروپا بسیار فعال بوده است.

در نهایت میخواهم از صدها کارمند (همچنین مشتری و بازدید کننده) گوگل که به
ارائه‌ها و کارگاه‌های من آمده‌اند تشکر کنم. بازخورد مثبت آنها به
پیش‌نمونه‌سازی، آزمایش‌های شخصیشان با آن و پیشنهادات و علاقه دائم آنها
من را بر این داشت که پیش‌نمونه سازی یک \textbf{چیز} \emph{درست} است.

\textbf{این کتاب به خانواده‌ام تعلق دارد:}

به پدرم که همیشه به من اعتماد داشت و از خودگذشتگی‌های بزرگی برای
سرمایه‌گذاری روی من انجام داده است. وقتی من شروع به کار کردم چیز جز
مجموعه‌ای از ایده‌های عجیب و غریب نداشتم. پدر متشکرم تو اولین سرمایه
گذار پر خطر من بودی!

به مادرم که اجازه داد من از ایتالیا در هنگامی که تنها ۱۷ سال داشتم خارج
شوم تا رویاهایم را در سیلیکون ولی پی بگیرم. مادر متشکرم، میدانم که این
تصمیم چقدر برات سخت بوده‌است.

به همسرم،که همیشه پشتیبان ریسک‌پذیری من در کارآفرینی بوده و خانه‌ی‌مان
به خانه زندگی مبدل کرده و بزرگ کردن فرزندان را آسان.

به فرزندانم، که مایه فخر هر روزه من هستند و مسئولیت پدر بودن را برای من
بسیار آسان کرده‌اند.
