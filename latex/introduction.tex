هم‌اینک، میلیون‌ها انسان در سراسر دنیا قلب، روح، امیدها، آرزوها، زمان،
پول و انرژی خود را صرف توسعه ایده‌های جدیدی می‌کنند که به محض راه‌اندازی
به شکل ناراحت‌کننده‌ای شکست می‌خورند.

همچنین در همین لحظه، تعداد بسیار کمتری در حال توسعه ایده‌های جدیدی هستند
که موفق خواهند شد. حتی برخی از آنها بسیار موفق شده و این ایده‌ها آیپاد
بعدی، گوگل بعدی، و توییتر بعدی خواهند بود.

شما در کدام گروه هستید؟

بسیاری بر این باورند که در حال کار روی محصول برنده هستند، اما می‌دانیم
که این موضوع نمی‌تواند درست باشد.

بیشتر ایده‌های جدید شکست می‌خورند و پیش‌بینی «موفقیت در بازار» هر ایده‌ی
جدیدی با هر درجه‌ای از اطمینان تقریبا غیرممکن است. بسیاری از ایده‌های
«\emph{شکست ناپذیر}» شکست‌های بزرگی از آب در می‌آیند. در حالی که برخی
ایده‌های جنون آمیز «\emph{کی اینو میخواد؟}» موفقیت‌هایی تماشایی می‌شوند.

ممکن است بعضی از افراد و برخی از شرکت‌ها توان بیشتری از دیگران در
پیش‌بینی موفقیت داشته باشند، اما بهترین سرمایه‌گذاران ریسک
پذیر\footnote{\lr{Venture} \lr{Capital}}، سرمایه گذاران و کارآفرینان
بطور مرتب سرمایه بیش از حدی روی \emph{ایده‌های غلط} گذاشته و مرتبا به
صورت فعال روی \emph{ایده درست} سرمایه گذاری نمی‌کنند.

اگر همه آن چیزی که ما داریم \emph{ایده‌ای} برای یک محصول (یا سرویس، یا
کتاب و موارد مشابه) جدید باشد، بهترین کاری که می‌توانیم انجام دهیم، جمع
آوری \emph{نظرات} در مورد کاربردی بودن و پتانسیل بازار آن ایده است.
ایده‌ها فازی\footnote{\lr{Fuzzy}} و انتزاعی هستند. نظرات ذهنی بوده و حتی
بیشتر از ایده‌ها انتزاعی هستند. وقتی شما این دو را با هم ترکیب می‌کنید
یک مجموعه بزرگ فازی از انتزاعات و نظرات دارید. چیز زیادی برای ادامه دادن
وجود ندارد.

نمونه‌های اولیه\footnote{\lr{Prototype}} می‌توانند به جای ایده‌ها و
نظرات، به تست و ارزیابی پتانسیل بازار یک ایده جدید به صورت درست و عینی
کمک کنند. اما در بسیاری از موارد، توسعه یک «نمونه اولیه مناسب» بسیار
سخت، پرهزینه و زمان‌بر است. صرف هفته‌ها، ماه‌ها یا سال‌ها زمان و صدها،
هزاران و حتی میلیون‌ها دلار برای توسعه یک نمونه اولیه کاملا عادی است.
همچنین، نمونه‌های اولیه برای پاسخ به سوالاتی مانند «آیا می‌توانیم این را
بسازیم؟» یا «این به همانگونه که مورد انتظار است عمل می‌کند» ساخته
می‌شوند و تاکیدی بر «آیا بایستی این را اصلا بسازیم؟» یا «اگر این را
بسازیم، آیا مردم آنرا می‌خرند و از آن استفاده می‌کنند؟» ندارند. اگر شما
بتوانید با نمونه اولیه به دو سوال آخر جواب مثبت بدهید، دو سوال اول از
درجه اهمیت کمی برخوردارند.

نمونه‌های اولیه به شما کمک می‌کنند که زودتر شکست بخورید، اما این شکست به
اندازه کافی سریع و کم هزینه نیست. هرچه بیشتر روی چیزی سرمایه گذاری کنید
سخت‌تر می‌توانید از آن دست کشیده و قبول کنید که این چیز غلط است. وقتی
شما یک «نمونه اولیه مناسب» داشته باشید، کمی بیشتر روی آن کار کردن و
بیشتر روی آن سرمایه گذاری کردن اغوا کننده است: «اگر ما این ویژگی را
اضافه کنیم، من مطمئنم مردم از آن استفاده خواهند کرد». نمونه‌های اولیه
معمولا تبدیل به \emph{محصولات اولیه} (نمونه‌های اولیه‌ای که روی آنها
زمان بیش از حدی گذاشته شده) می‌شوند و معمولا شما یک شکست سریع را تجربه
می‌کنید.

مرحله میانی بین ایده‌های انتزاعی و «نمونه‌‌ی اولیه مناسب» \emph{پیش
نمونه}\footnote{\lr{Pretotype}} است. پیش نمونه‌ها امکان جمع آوری اطلاعات
ارزشمند مربوط به نحوه استفاده و بازار را برای شروع و یا عدم شروع یک ایده
جدید فراهم می‌کنند. این اطلاعات به کمک پیش‌نمونه‌ها در کسری از هزینه
نسبت به نمونه‌های اولیه بدست می‌آید: ساعت‌ها یا روزها به جای هفته‌ها یا
ماه‌ها و چند پنی بجای چند دلار. پیش نمونه‌ها به شما کمک می‌کنند که به
سرعت شکست خورده و سریع بهبود بیابید. این سریع شکست خوردن زمان، پول،
انرژی و اشتیاق کافی برای کاوش ترفند‌ها و ایده‌های جدید در اختیار شما
قرار می‌دهد، تا زمانی که ایده‌ای بیابید که به نظر موافق طبع مردم است؛
همان «\emph{یک} \textbf{چیز} \emph{درست}» نادر و شگفت انگیز.

بسیاری از مواردی که در این کتاب می‌خوانید به نظر شما واضح می‌آید. اما
قبل از عبور سریع از روی آنها، نگاهی به محصولات، سرویس‌ها، نرم‌افزارها،
کتاب‌ها و \ldots{} اطراف خود بیاندازید که هرروز ارائه شده و خیلی زود هم
شکست می‌خورند. دلیل شکست اکثر این محصولات این نیست که افرادی که آنها را
تولید کرده‌اند نادان، تنبل یا بی‌کفایت بوده‌اند. همچنین این شکست به دلیل
کیفیت پایین ساخت محصولات و بازاریابی آنها نیست. این شکست بخاطر درست
نبودن محصولی است که آنها کار را با آن شروع کرده‌اند.

در صورتی که کار خود را به تازگی شروع نکرده باشید، این احتمال وجود دارد
که شما به گذشته کاری خود و محصولاتی که روی آن کار کرده‌اید نگاهی انداخته
و تشخیص بدهید که گذشت زمان معلوم ساخته است، برخی از آنها محصولات درستی
نبوده‌اند. این دقیقا در مورد من صدق می‌کند. من شانس کار کردن روی
محصولاتی را داشته ام که ماه‌ها کار را تبدیل به میلیونها(حتی
میلیاردها)دلار کرده و همچنین روی محصولاتی که سالها کار و ده‌ها میلیون
دلار را تبدیل به «خاکستر» کرده است.

با وجود اینکه این نسخه کتاب به خودی خود یک پیش نمونه است(من بعدا در این
مورد بیشتر توضیح خواهم داد)، بایستی ارزش کافی برای وقت شما را داشته
باشد. من خالصانه از این حقیقت که شما این کتاب را میخوانید قدردانی میکنم.
لطفا نظرات خود را برای من بفرستید - من نیاز به اطلاعات برای فهمیدن درستی
سرمایه‌گذاری برای تبدیل این پیش کتاب به یک کتاب مناسب دارم.

با تشکر از شما

آلبرتو ساویا( \lr{asavoia@gmail.com} ) آگوست 2011

ترجمه: عباس یزدان پناه بهار ۱۳۹۴
