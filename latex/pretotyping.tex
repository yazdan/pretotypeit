\section{پیش‌نمونه سازی
چیست}\label{ux67eux6ccux634ux646ux645ux648ux646ux647-ux633ux627ux632ux6cc-ux686ux6ccux633ux62a}

اکنون که شما بصورت اولیه میدانید که منظور من از آن \textbf{چیز}
\emph{درست} چیست، ما می‌توانیم مقدمه قابل قبول در باره پیش‌نمونه سازی
داشته باشیم. بهترین راه برای این کار بررسی دو داستانی است که من را به
فکر این موضوع انداخت: «آزمایش» تبدیل متن به گفتار آی بی ام و «آزمایش»
پالم پایلوت.

\section{آزمایش متن به تبدیل متن به گفتار آی بی
ام}\label{ux622ux632ux645ux627ux6ccux634-ux645ux62aux646-ux628ux647-ux62aux628ux62fux6ccux644-ux645ux62aux646-ux628ux647-ux6afux641ux62aux627ux631-ux622ux6cc-ux628ux6cc-ux627ux645}

اولین بار من این داستان را چند سال پیش در یک ارائه در یکی از کنفرانس‌های
نرم‌افزار شنیدم. من دقیقا نمی‌دانم تعاریف من از ماجراها چقدر دقیق است.
ممکن است من برخی از جزئیات را اشتباه دریافته باشم، اما نتیجه اخلاقی
ماجرا بسیار از جزئیات مهم‌تر است. با در نظر گرفتن این ایراد، این داستانی
است که من بخاطر می‌آورم.

چند دهه پیش، قبل از عصر اینترنت و حتی قبل از طلوع کامپیوترهای شخصی، آی
بی ام بخاطر ماشین تحریر و کامپیوترهای مینفریمش مشهور بود. در آن زمان
تایپ‌کردن یکی از ویژگی‌هایی بود که افراد کمی آنرا بخوبی انجام می‌دادند
که بیشتر آنها منشی، نویسنده و برخی از برنامه‌نویسان بودند. بیشتر افراد
از یک انگشت برا تایپ استفاده می‌کردند که کند و ناکارآمد بود.

آی بی ام درست در نقطه‌ای قرار داشت که بتواند از تجربه خود در بازار
کامپیوتر و ماشین تحریر استفاده کرده یک ماشین تبدیل متن به گفتار توسعه
دهد. این ابزار به افراد اجازه میداد که در یک میکروفن صحبت کنند و متن
بصورت «جادویی» روی صفحه نمایش ظاهر شود و دیگر نیازی به تایپ کردن نباشد.
این دستگاه پتانسیل زیادی برای کسب درآمد برای آی بی ام داشت و \emph{ریسک
بزرگ} روی این موضوع برای شرکت قابل قبول به نظر میرسید.

اما در این میان چندین اشکال بزرگ وجود داشت. کامپیوترها در آن زمان کم
قدرت تر و بسیار گرانتر از امروزه بوده و تبدیل متن به گفتار نیاز به
پردازش زیادی داشت. همچنین، با داشتن قدرت محاسباتی کافی، تبدیل متن به
گفتار یک مساله بسیار سخت علوم کامپیوتر بوده و هست. دست و پنچه نرم کردن
با این مساله نیاز به سرمایه‌گذاری عظیم -حتی برای آی بی ام- و سال‌های
زیاد برای تحقیق داشت. اما همه به این دستگاه نیاز داشتند. این یک موفقیت
واضح خواهد بود. یا اینطور خواهد شد؟

برخی در آی بی ام توسط افرادی که میگفتند به مردم تبدیل متن به گفتار «نیاز
داشته و قطعا آنرا خریداری نموده و استفاده میکنند» قانع نشده بودند و فکر
نمی‌کردند این دستگاه به موفقیت برسد. آنها از این می‌ترسیند که سال‌ها
تحقیق و سرمایه شرکت صرف توسعه دستگاهی شود که اندکی آنرا میخرند که این یک
فاجعه در کسب و کار است. به زبان پیش‌نمونه سازی آنها مطمئن نبودند که
تبدیل گفتار به متن یک \textbf{چیز} \emph{درست} است. همچنین، مردم تا کنون
از تبدیل گفتار به متن استفاده نکرده بودند ، پس آنها نمی‌توانستند بصورت
قطعی بدانند که کسی به این دستگاه نیاز دارد یا نه؟ آی بی ام نیاز به بررسی
قابلیت ماندگاری این دستگاه در کسب و کار را داشت اما ساختن حتی یک نمونه
اولیه نیاز به سال‌ها زمان داشت. آنها بجای آن یک آزمایش مبتکرانه طراحی
کردند.

آنها مشتریان بالقوه دستگاه تبدیل گفتار به متن خود را که به نظر آنها قطعا
خریدار این دستگاه بود در اتاقی با یک کامپیوتر، یک میکروفن و یک صفحه
نمایش بدون کیبرد قرار دادند. به آنها گفتند که یک ماشین تبدیل خودکار
گفتار به متن ساخته‌اند و میخواهند ارزیابی کنند که آیا مردم از استفاده از
آن لذت میبرند یا نه. وقتی آزمایش دهنده‌ها شروع به صحبت در میکروفن کردند
متن آنها تقریبا بی درنگ و بدون خطا روی صفحه نمایش ظاهر می‌شد! کاربران
تحت تاثیر قرار گرفته بودند. این برای واقعی بودن خیلی خوب بود که معلوم شد
نبوده است.

اتفاق پشت صحنه که این آزمایش را مبتکرانه میکند این بود که ماشین تبدیل
متن به گفتار حتی یک نمونه اولیه نبود. کامپیوتر موجود در اتاق خالی ساختگی
بود. در اتاق کناری یک تایپیست کارآزموده در حال گوش کردن به صدای کاربر
بود و با استفاده از کیبرد صحبت‌های او را تایپ و دستورات او را اجرا
میکرد. هرچه تایپیست تایپ میکرد روی صفحه نمایش کاربر نشان داده می‌شد.
صحنه سازی انجام شده به گونه‌ای بود که کاربر قانع میشد که خروجی روی صفحه
نمایش خروجی دستگاه تبدیل گفتار به متن است.

اما آی بی ام از این آزمایش چه یاد گرفت؟

این چیزی است که من شنیده‌ام: بعد از تاثیر اولیه بوسیله «تکنولوژی»،
بسیاری از افرادی که خریدار این سیستم بودند پس از چند ساعت کار با این
سیستم نظرشان عوض شد. گفتن چندین خط متن از طریق گفتار در کامپیوتر حتی با
استفاده تبدیل تقریبا بدون نقص و سریع توسط تایپیست هم دارای مشکلات زیادی
بود: گلوی افراد بر اثر صحبت زیاد خشک میشد، محیط کار پر از هم همه میشد و
به درد موارد محرمانه نمی‌خورد.

براساس نتایج این آزمایش، آی بی ام باز هم در تبدیل گفتار به متن سرمایه
گذاری نمود اما در مقیاسی به مراتب کمتر - آنها رو \emph{اعتبار شرکت قمار}
نکردند.

اینطور به نظر میرسد که این یک تصمیم درست در کسب و کار بوده است. کیبردها
نشان‌داده اند که در مورد وارد کردن متن به سختی شکست می‌خورند. سی سال پیش
مردم نمی‌توانستند تایپ کنند اما اکنون در هر دفتر (یا کافی شاپی) افراد
مختلف در سنین و شغل‌های مختلف را می‌بینید که در حال تایپ روی لپ‌تاپ‌های
خود هستند. در دستگاه‌هایی که کیبرد با سایز استاندارد غیر قابل استفاده
است همانند موبایل‌ها، تبدیل متن به گفتار میتواند یک \textbf{چیز}
\emph{درست} باشد اما در غیر اینصورت هنوز بایستی کیبرد را شکست بدیهد.
کیبرد قطعا یک \textbf{چیز} \emph{درست} است.

راهبرد آی بی ام مبتکرانه بود، اما شما به آن چه عنوانی می‌دهند. صحنه سازی
تبدیل گفتار به متن به کمک یک تایپیست قطعا یک «نمونه اولیه مناسب» نیست
مگر اینکه قصد داشته باشید که واقعا یک تایپیست زنده را درون یک کامپیوتر
جا بدهید. آنها یک نمونه اولیه از تبدیل متن به گفتار نساختند، بلکه
\emph{وانمود} کردند که یک نمونه اولیه تبدیل متن به گفتار داشته و از آن
به منظور دریافت عکس‌العمل مشتری به محصول استفاده کردند. در این حالت آنها
امکان جمع آوری اطلاعات با ارزش بازار را براساس استفاده واقعی به جای نظر
افراد داشتند، همچنین سرمایه‌گذاری مالی و زمانی کمی انجام دادند.

به نظر من این راهبرد بسیار ارزشمند و جالب است، و این روش به اندازه کافی
از ساختن نمونه اولیه متفاوت بوده که نام خاص خودش (که بیشتر در مورد آن
صحبت خواهم کرد) و ارزش بررسی را دارد. اما اول از هم سعی به یافتن
مثال‌های دیگر در این زمینه کردم که یک مثال عالی پیاده کردم.
