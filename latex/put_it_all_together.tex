بالاخره تمام قطعات را جمع آوری کردیم، پس حالا میتوانیم چند مثال از ساختن
و تست پیش‌نمونه‌ها را بررسی کرده و براساس آنها تصمیم بگیریم. هنگام
خواندن این مثال‌ها از اینکه راه‌های دیگری برای پیش‌نمونه‌سازی این
ایده‌ها و تست آنها به ذهنتان برسد متعجب نشوید، زیرا یک راه برتر برای
این‌کار وجود ندارد. اگر راه‌های دیگری برای پیش‌نمونه سازی به ذهنتان نرسد
برای من جای تعجب دارد.

\section{مثال ۱: یک مشاهده گر سنجاب
کارکشته}\label{ux645ux62bux627ux644-ux6ccux6a9-ux645ux634ux627ux647ux62fux647-ux6afux631-ux633ux646ux62cux627ux628-ux6a9ux627ux631ux6a9ux634ux62aux647}

بیایید مثال خود را با پیش‌نمونه در جعلی بسازیم. همانطور که ممکن است به
یاد بیاورید، سندی به فکر نوشتن کتابی در مورد مشاهده سنجاب‌ها بود. از
آنجایی که نوشتن کتاب \emph{یک مشاهده گر سنجاب کار کشته} ماه‌ها زمان را
به خود اختصاص خواهد داد و او را از مشاهده‌ی سنجاب‌ها باز خواهد داشت، این
ایده خوبی است که کتاب را پیش‌نمونه سازی کند.

در مورد سندی، موفقیت کتاب تنها وابسته به تعداد افرادی است که کتاب را
میخرند(و به خرید مجدد آنها وابسته نیست) پس پیش‌نمونه‌سازی به منظور به
دست آوردن میزان علاقه اولیه کافی خواهد بود. پیش‌نمونه در جعلی برای این
حالت ایده‌آل خواهد بود. سندی اینگونه می‌تواند این کار را انجام دهد:

با ۱۰ دلار او می‌تواند دامنه این کتاب(
\lr{thecompeletesquirrelwatcher.com} ) را بخرد و یک صفحه اولیه حاوی
محتوای زیر بسازد:

\begin{quote}
علاقه مندان عزیز به سنجاب!!!
\end{quote}

\begin{quote}
از شما به خاطر علاقه‌ی‌تان به \emph{یک مشاهده گر سنجاب کار کشته} متشکرم.
من در حال کار روی این کتاب هستم، اما کتاب هنوز برای انتشار آماده نیست.
\end{quote}

\begin{quote}
برای رزور کردن یک نسخه از این کتاب با نرخ ویژه ۹/۹۸ دلار ایمیلی به
\lr{iwantthebook@thecompeletesquirrelwatcher.com} بفرستید.
\end{quote}

\begin{quote}
هنگامی که کتاب آماده شد در اولین فرصت به شما خبر خواهم داد.
\end{quote}

\begin{quote}
قیمت کتاب ۹/۹۸ دلار خواهد بود.
\end{quote}

\begin{quote}
در این زمان، اوقات خوشی در مشاهده سنجاب‌ها داشته باشید و حواستان به
واکسن هاری باشد!
\end{quote}

\begin{quote}
سندی(دختر سنجاب) واتسون
\end{quote}

همچنین او تبلیغی تحت وبی به شکل زیر ایجاد میکند

\begin{quote}
\textbf{آیا شما به مشاهده سنجاب‌ها علاقه دارید؟}
\end{quote}

\begin{quote}
\lr{thecompeletesquirrelwatcher.com}
\end{quote}

\begin{quote}
کتابی برای مشاهده‌گران سنجاب حرفه‌ای
\end{quote}

\begin{quote}
نوشته شده توسط سندی واتسون. تنها ۹/۹۸ دلار
\end{quote}

با خرج کردن چند دلار، او می‌تواند تبلیغ خود را در سایت‌هایی که به
سنجاب‌ها اختصاص یافته نشان دهد یا برای جستجوهای کلمات مرتبط با سنجاب به
نمایش گذاشته شود. وقتی افراد روی تبلیغ او کلیک می‌کنند بصورت اتوماتیک به
سایت او انتقال می‌یابند.

این پیش‌نمونه در جعلی کمتر از ۵۰ دلار هزینه داشته و نیاز به چند ساعت کار
دارد. این کار نیازی به تخصص خاصی ندارد.

هنگامی که این پیش‌نمونه ایجاد شد، سندی می‌تواند یک ماه یا بیشتر صبر کند.
بعد از این زمان او می‌تواند داده‌هایی که از سرویس تبلیغات آنلاین بدست
آورده است را تحلیل کند.

\begin{quote}
تعداد افرادی که تبلیغ را دیده‌اند: ۲۳۴۰۲ نفر
\end{quote}

\begin{quote}
تعداد افرادی که روی تبلیغ کلیک کرده‌اند: ۶۳۴ نفر
\end{quote}

\begin{quote}
افرادی که ایمیل زده‌اند و گفته اند که کتاب را می‌خرند: ۲۳۰ نفر
\end{quote}

در اینجا چندین معیار میزان علاقه اولیه جالب قابل محاسبه است.

اولین معیار نشان‌دهنده این است که چند نفر سنجاب دوست به اندازه کافی
علاقه‌مند هستند تا روی تبلیغات مربوط به کتاب در مورد مشاهده سنجاب کلیک
کنند. اولین معیار میزان علاقه اولیه بصورت زیر محاسبه می‌شود:

\begin{quote}
اولین میزان علاقه اولیه = تعداد کلیک‌های روی تبلیغ / تعداد نمایش‌های
تبلیغ
\end{quote}

در این حالت این مقدار برابر ۶۳۴/۲۳۴۰۲ تقریبا ۲/۷ درصد است.

دومین معیار میزان علاقه اولیه درصد افرادی است که بعد از کلیک کردن رو
تبلیغ به اندازه کافی علاقه‌مند هستند که به سندی ایمیل بز‌نند.

\begin{quote}
دومین میزان علاقه اولیه = تعداد ایمیل‌ها / تعداد مشاهده‌های صفحه اول
سایت
\end{quote}

در این حالت برابر ۳۵ درصد(۲۳۰/۶۳۴) است.

این یک عدد بسیار امیدوار کننده است. ۳۶ درصد افرادی که سایت سندی را
مشاهده‌کرده اند گفته‌اند که یک نسخه از کتاب سندی را می‌خواهند. درست است
که همه آنها کتاب را نخواهند خرید اما این عدد بسیار خوب است.

حال نوبت به تصمیم‌گیری دشوار میرسد: با توجه به این داده‌ها آیا سندی به
نوشتن کتاب بپردازد یا نه؟

این به انتظار سندی از کتاب وابسته است. داده‌ها می‌گوید که این کتاب خیلی
بعید است که به لیست کتاب‌های پرفروش نیویورک تایمز وارد شود بخاطر اینکه
تعداد افراد علاقه مند به سنجاب‌ها چندان نیستند. او به دنبال متخصص شدن و
مرجع شدن در این حوزه است و همچنین می‌خواهد با فروش چند صد نسخه کتاب در
سال مخارج سفرهای مشاهده‌ی سنجاب خود را تامین کند. در این حالت اطلاعات به
او می‌گویند که \emph{یک مشاهده گر سنجاب کار کشته} احتمالا یک
\textbf{چیز} \emph{درست} برای تعداد کافی از آدم‌هاست تا سندی را خوشحال
کند.

\section{\texorpdfstring{مثال دوم: نرم افزار باب با اسم \emph{رتبه
بشقاب}}{مثال دوم: نرم افزار باب با اسم رتبه بشقاب}}\label{ux645ux62bux627ux644-ux62fux648ux645-ux646ux631ux645-ux627ux641ux632ux627ux631-ux628ux627ux628-ux628ux627-ux627ux633ux645-ux631ux62aux628ux647-ux628ux634ux642ux627ux628}

برای مثال، فرض کنید که باب متخصص تغذیه است که می‌خواهد نرم‌افزار موبایلی
بنویسید که با تحلیل عکس یک وعده‌ی غذایی میزان ارزش آن وعده را به همراه
یک امتیاز به کاربران اطلاع دهد. امتیاز به عنوان مثال می‌تواند «الف: سالم
و ارزشمند» تا «و: هله، هوله» باشد. بگذارید این \textbf{چیز} باب را
\emph{رتبه بشقاب} بنامیم.

باب در مورد این نرم‌افزار با دوستان و افراد دیگر صحبت می‌کند، و بیشتر
آنها به او می‌گویند که این یک ایده‌ی عالی است و آنها قطعا از آن استفاده
خواهند کرد. خوشبختانه باب در مورد سرزمین فکر شنیده و می‌داند که نظرات
چقدر می‌تواند گمراه‌کننده باشد. او به قطع نمی‌داند که چه افرادی از این
نرم‌افزار استفاده کرده و حاضرند برای آن هزینه کنند. آیا کاربران به یاد
خواهند داشت که چند لحظه تامل کرده و عکسی از غذای خود قبل از خوردن آن
بگیرند؟ آیا آنها برای مدت محدودی به عنوان سرگرمی از آن استفاده کرده و
سپس آنرا فراموش خواهند کرد؟

باب همچنین می‌داند که توسعه یک سیستم نرم‌افزاری که واقعا بصورت اتوماتیک
یک وعده‌ی غذایی را براساس عکس آن تحلیل کند قطعا کار و هزینه بسیاری خواهد
برد و همچنین می‌داند رسیدن به نقطه‌ای که این نرم‌افزار به اندازه کافی
خوب و دقیق کار کند ممکن است غیر ممکن باشد(همانند مساله تبدیل گفتار به
متن آی بی ام)

مسائل باز بسیاری که بایستی پاسخی برای آنها یافت شود وجود دارد و تکنولوژی
آنها بسیار پر هزینه است. قطعا این \textbf{چیز} نیازمند پیش‌نمونه سازی
است.

\subsection{قدم اول: پیش‌نمونه‌های در جعلی و
پینوکیو}\label{ux642ux62fux645-ux627ux648ux644-ux67eux6ccux634ux646ux645ux648ux646ux647ux647ux627ux6cc-ux62fux631-ux62cux639ux644ux6cc-ux648-ux67eux6ccux646ux648ux6a9ux6ccux648}

تا الان شما نباید از اینکه من به عنوان اولین قدم در جعلی را پیشنهاد
داده‌ام متعجب باشید. باب بایستی به گونه‌ای در جعلی بسازد تا میزان علاقه
اولیه را اندازه گیری نماید(برای این منظور به مثال قبل مراجعه کنید).

بیایید فرض کنیم که داده‌های میزان علاقه اولیه امیدوار کننده است. اما،
چشم انداز و تعریف باب از موفقیت این نرم‌افزار علاوه بر علاقه اولیه
استفاده مداوم است(به عنوان مثال میزان علاقه مداوم ترغیب کننده). اگر
انجام آنچه نرم‌افزار نیازمند آن است سخت و عذاب آور باشد، افراد از انجام
آن سرباز خواهند زد. اصلا خود باب آنرا انجام خواهد داد؟ آیا باب به یاد
خواهد آورد که از غذایش قبل از شروع آن عکس بگیرد؟ آیا او از انجام اینکار
در حضور دیگران(خصوصا در رستوران‌ها) خجالت زده خواهد شد؟ آیا او تنها از
غذاهای سالم خود عکس خواهد گرفت و به راحتی دسر بستی موزی خود را فراموش
خواهد کرد؟

اگر خود ما به \textbf{چیز}مان ایمان نداشته و از آن استفاده نکنیم، چگونه
می‌توانیم دیگران را خالصانه راضی کرده یا انتظار داشته باشیم که آنها این
کار را انجام خواهند داد. برای پاسخ به این سوال، باب بایستی راهی که جف
هاوکینز برای پیش‌نمونه‌سازی پالم پایلوت طی کرده است را دنبال کند. بایستی
از یک پیش‌نمونه پینوکیو برای تست این ایده بصورت شخصی استفاده کند. از
آنجایی که باب تلفن هوشمندی دوربین‌دار دارد او نیازی به رفتن به کارگاه و
ساختن یک بلوک چوبی ندارد. او به سادگی می‌تواند وانمود کند که نرم‌افزار
دوربین تلفن او همان نرم‌افزاری است که او علاقه‌مند به ساختن آن است. او
جاهای خالی را با تخیلات خود پر خواهد کرد.

بعد از چند روز استفاده از پیش‌نمونه پینوکیو، باب در می‌یابد که علاقه
اولیه او رو به افول گذاشته و او عکس‌های کمتر و کمتری می‌گیرد، پس ممکن
است مشکلی وجود داشته باشد. البته او می‌تواند عذرهایی برای این شکست
بیاورد «این نرم‌افزار برای من نیست، برای مشتریان من است، من میدانم که چه
باید بخورم و من به آن نیازی ندارم». او ممکن است در این حالت خاص درست
بگوید اما این مورد هنوز جای نگرانی دارد. بحث در مورد «من از آن استفاده
نمی‌کنم، اما بقیه می‌کنند» یک پرچم قرمز بزرگ است که روی همه جای آن یک
\textbf{چیز} غلط نوشته است و به این راحتی نمی‌توان از کنارش گذشت.

به منظور ادامه مثال، بیایید فرض کنیم که باب آنقدر سریع به گرفتن عکس از
غذایش عادت می‌کند که این یک عادت برای او شده و او آنرا بصورت اتوماتیک و
مدوام انجام می‌دهد. علاوه بر اینها، وقتی او اینکار را در جلوی دیگران
انجام می‌دهد آنها در این باره از او می‌پرسند و به او می‌گویند که از چنین
نرم‌افزاری استقبال می‌کنند. همچنین او عکس‌ها را در یک آلبوم آنلاین به
نمایش گذاشته تا بتواند غذاهایی را که خورده است پیگیری کرده و آنها را به
دوست متخصص تغذیه‌اش بفرستد تا نظر او را نیز در مورد رژیم غذایی‌اش داشته
باشد. این نشانه‌ی خوبی است. باب اکنون می‌داند که او خودش بصورت مداوم از
نرم‌افزار استفاده خواهد کرد و او آنرا به اندازه کافی کارا دانسته که چند
«ویژگی» جدید را برای آن «پیاده‌سازی» کند(مثل پست کردن عکس‌ها در یک آلبوم
آنلاین و ارسال آن به دوست متخصصش).

تست دو پیش‌نمونه اول او خوب از کار در آمد، میزان علاقه اولیه خوب بود و
میزان علاقه مداوم شخصی او نیز خیلی خوب بود. حالا نوبت به این رسیده که
ببیند افراد دیگری بطور مداوم از این نرم‌افزار استفاده خواهند کرد.

باب نیاز به سنجش میزان علاقه مدوام داشته و پیش‌نمونه در جعلی و حتی
پیش‌نمونه ساده پینوکیو به این موضوع کمکی نخواهند کرد(پیش‌نمونه پینوکیو
نیاز به تخیل بزرگ و وانمود کردن در مورد ویژگی‌ها و کارکردها دارد، آنها
برای قانع کردن سازنده بسیار خوب هستند اما به درد جمع آوری اطلاعات
کاربران نمی‌خورند) . چیزی که باب به آن نیازمند است یک پیش‌نمونه ساده ولی
کارا است. متاسفانه باب متخصص تغذیه است و از برنامه نویسی سر در نمی‌آورد.
قبل از استخدام یک برنامه نویس، آیا راه سریع‌تر و ارزان‌تری که او تخمینی
از علاقه مدوام به دست بیاورد وجود ندارد؟ بله که وجود دارد!

\subsection{پیش‌نمونه مفت با فن‌آوری پایین ترک
میکانیکی}\label{ux67eux6ccux634ux646ux645ux648ux646ux647-ux645ux641ux62a-ux628ux627-ux641ux646ux622ux648ux631ux6cc-ux67eux627ux6ccux6ccux646-ux62aux631ux6a9-ux645ux6ccux6a9ux627ux646ux6ccux6a9ux6cc}

از آنجایی که باب یک متخصص تغذیه است حدود ۵۰۰ مشتری داشته و او می‌تواند
از بخش کوچکی از مشتریانش(مثلا ۵۰ نفر در حدود ۱۰ درصد) بخواهد که آیا آنها
علاقه‌مند به شرکت در یک آزمایش به مدت یک ماه هستند یا نه. تنها کاری که
آنها باید انجام دهند این است که قبل از خوردن عکسی از غذای خود گرفته و
آنرا برای او ایمیل کنند. در جواب او هر روز سطح کیفیت غذاهای آن وعده و
چند توضیح و پیشنهاد در مورد چگونگی بهبود رژیم آنها برایشان میفرستد. هیچ
چیز عجیب‌غریب یا زمان‌بری وجود ندارد. چیزی شبیه این‌ها

\begin{quote}
ماری عزیز
\end{quote}

\begin{quote}
باتشکر از کمک شما برای تست رتبه بشقاب
\end{quote}

\begin{quote}
رتبه بندی شما به این شکل است:
\end{quote}

\begin{quote}
صبحانه: و(تخم مرغ و بیکن، تو بهتر از این می‌تونی عمل کنی)
\end{quote}

\begin{quote}
ناهار: ب (سالاد خوب است، سس سالاد مایونز بد است)
\end{quote}

\begin{quote}
شام: الف منفی (مرغ و سبزیجات سالم به نظر می‌رسند، اما منفی بخاطر آن نان
کره‌ای است)
\end{quote}

\begin{quote}
سعی کن برای وعده‌ها آتی میوه و سبزیجات مصرف کنی
\end{quote}

\begin{quote}
باتشکر
\end{quote}

\begin{quote}
باب
\end{quote}

بایید فرض کنیم که ۳۰ نفر(از ۵۰ نفر) مشتریان باب با انجام آزمایش موافقت
می‌کنند(میزان علاقه اولیه ۳۰/۵۰ یا ۶۰ درصد است). در ابتدا باب ناامید
می‌شود زیرا با اینکه میزان علاقه اولیه بسیار زیاد بود اما او انتظار داشت
تمام مشتریانش یا حداقل ۸۰ یا ۹۰ درصد از آنها علاقه‌مند به انجام این
آزمایش باشند. بعد از حرف زدن با مشتریانی که علاقه‌مند به شرکت در این کار
نبودند او از مواردی آگاهی پیدا کرد که قبلا به ذهنش نرسیده بود. به عنوان
مثال بسیاری از مشتریانش، موبایلی که به اینترنت متصل باشد نداشتند به همین
خاطر نمی‌توانستند عکس‌ها را به او ایمیل کنند. اندکی از آنها هم از به
اشتراک گذاشتن عکس واقعی غذایشان با او یا کسان دیگر احساس ناراحتی
می‌کردند اما مشکلی با تحلیل اتوماتیک غذایشان توسط کامپیوتر نداشتند. این
موارد بسیار ارزشمند بوده و باید در ادامه مسیر بخاطر سپرده شوند.

هنگام شروع آزمایش باب به ۳۰ مشتری دواطلب مراحل انجام کار را ارسال
می‌کند(عکسی از هر چیزی که میخورید بگیرید و آنرا به ایمیل من بفرستید).
هنگامی که ایمیل‌ها به دست او میرسند (۸۰ ایمیل در روز) او آنها را بررسی
کرده و نتایج تحلیل تغذیه را به آنها میفرستد. این کار بسیار زیادی است اما
از آنجایی که او برنامه‌نویس نیست این کار برای او آسانتر و کم هزینه‌تر
است.

بعد از یک ماه اجرا این طرح جدول میزان علاقه مدوام به شکل زیر است:

\begin{longtable}[c]{@{}lll@{}}
\toprule
هفته & افراد فعال از ۳۰ نفر & تعداد عکس‌های ارسالی\tabularnewline
\midrule
\endhead
۱ & ۲۸ & ۲۳۴\tabularnewline
۲ & ۲۴ & ۱۹۸\tabularnewline
۳ & ۲۲ & ۱۶۸\tabularnewline
۴ & ۲۲ & ۱۷۲\tabularnewline
\bottomrule
\end{longtable}

همانگونه که همیشه اتفاق می‌افتدبرخی از افراد که در ابتدا گفته‌بودند
علاقه‌مند به این کار هستند اصلا عکسی نمی‌فرستند و با گذشت زمان دواطلبان
اولی پشیمان می‌شوند. در انتهای ماه اما هنوز دو سوم دواطلبان بصورت فعالی
عکس ارسال می‌کنند. این امیدوارکننده است.

بسیاری از کاربران از او میخواند که ویژگی‌ها و کارکردهای جدیدی به
نرم‌افزار اضافه کند. «باب آیا می‌توانی معدل من را برای من بفرستی؟» «اگر
من فراموش کنم که عکس غذا را بگیرم آیا می‌توانم که توصیف غذا را برایت
ارسال کنم؟» «آیا می‌توانی به من منویی بفرستی که در تمام روز بیشترین
امتیاز(الف) بگیرم؟» که این به میزان امیدواری می‌افزاید.

از سوی دیگر برخی شکایت‌هایی دارند: «باب موبایل من در کافه تریا خوب خط
نمی‌دهد و من برای ارسال ایمیل مجبورم به بیرون از کافه بروم تا ایمیل
ارسال کنم در حالی که غذایم در حال سرد شدن است.»

وقتی شما از کاربرانتان بازخوردی دریافت نمی‌کنید به احتمال زیاد آنها یا
اصلا از محصول شما استفاده نمی‌کنند یا به آن به اندازه کافی اهمیت
نمی‌دهند که بازخورد خود را در مورد بهبود یا بهتر کردن آن ارائه کنند.
دریافت بازخورد خوب یا بد یک نشانه بزرگ است. آنها به اندازه کافی اهمیت‌
می‌دهند که پیشنهاد داده یا شکایت کنند.

اوضاع باب به نظر خوب می‌رسد: میزان سطح علاقه مداوم قوی و بازخوردهای زیاد
کاربران. ایده \emph{رتبه بشقاب} باب ممکن است یک \textbf{چیز} \emph{درست}
باشد.

هنوز مساله درآمد و سودآوری وجود دارد. باب می‌خواهد از اینکار کسب و کاری
راه‌اندازی نماید. آیا افرادی که از این نرم‌افزار بصورت مجانی استفاده
می‌کردند حاضرند برای این سرویس هزینه‌ای پرداخت کنند؟ آنها چقدر حاضرند
پرداخت کنند: ۱۰ دلار در ماه شاید ۳۰ دلار در ماه؟ تا الان من مطمئنم که
میدانید که بایستی چگونه به این سوال پاسخ دهید. باب هنوز ۴۵۰ مشتری دیگر
برای آزمایش دارد. او می‌تواند از ۱۰۰ نفر آنها بپرسد که برای این سرویس با
هزینه ۱۰ دلار در ماه ثبت نام کنند و از ۱۰۰ نفر دیگر بپرسید آیا آنها ۳۰
دلار در ماه هزینه خواهند کرد تا بتواند میزان علاقه اولیه و مداوم را در
دوحالت اندازه بگیرد.

تنها چند تن از مشتریانش برای سرویس ۳۰ دلار در ماه ثبت نام کرده‌اند و ۴۲
نفر از مشتریانش ۱۰ دلار در ماه را پذیرفته‌اند. این عدد بیش از آن چیزی
است که او می‌تواند بصورت دستی آنرا انجام دهد. الان زمان آن رسیده که روی
اتوماسیون سرمایه گذاری کند. متاسفانه او به این نتیجه رسید که تکنولوژی
تحلیل اتوماتیک ظرف غذا براساس تصویر حداقل چند سال نیاز به زمان دارد. اما
او دریافته است که دانشجویان سال آخر می‌توانند با ساعتی ۱۵ دلار کار تحلیل
را بخوبی او انجام دهند. او با یک حساب و کتاب ساده به این نتیجه رسید که
او می‌تواند ۴ دلار به ازای هر مشتری در ماه سود کند.

بعد از چند ماه اجرای سرویس برای مشتریانش به این شکل و سود آفرینی، باب
تصمیم می‌گیرد که بزرگ عمل کند (\textbf{چیز} او یک \textbf{چیز}
\emph{درست} است). او یک برنامه نویس استخدام می‌کند تا یک نرم‌افزار خاص
منظوره (بجای پیش‌نمونه سنتی ایمیل زدن) برای او طراحی کند و دانشجویان
بیشتری را برای انجام این تحلیل در سطح وسیع‌تر آموزش دهد.

نرم افزار \emph{رتبه بشقاب} باب یک \textbf{چیز} \emph{درست} بود و به
همین خاطر تعداد افراد سالم‌تری وجود دارند.

آیا شما از پایان خوش بدتان می‌آید؟
